\documentclass{article}
\usepackage[russian]{babel}
\usepackage[letterpaper,top=2cm,bottom=2cm,left=3cm,right=3cm,marginparwidth=1.75cm]{geometry}
\usepackage{wrapfig}
\usepackage{indentfirst}
\usepackage{amsmath}
\usepackage{graphicx}
\usepackage[colorlinks=true, allcolors=blue]{hyperref}

\title{Теорема о промежуточной оси}
\author{Соложенкин Севастьян \\ студент МФТИ 1 курс, группа Б01-411}
\date{Январь 2025}

\begin{document}
\maketitle

\section{Введение}
Данная работа подготовлена как вопрос по выбору для экзамена по общей физике 1 семестра МФТИ. Работа посвящена рассмотрению вращения твёрдого тела по инерции вокруг неподвижной точки, а именно теореме о промежуточной оси. Теорема заключается в том, что свободоное вращение твёрдого тела около главной оси с промежуточным моментом инерции неустойчиво, в отличие от вращений около главных осей с эктремальными моментами инерции. Все термины будут объяснены далее.  
\\
\par
\textbf{План работы:}
\begin{itemize}
    \item Предварительная теория, необходимая для понимания материала работы
    \item Теоретическая часть, заключающаяся в приведении двух вариантов доказательства теоремы о промежуточной оси.
    \item Теоретическая справка об эффекте Джанибекова. Описание установки для демонстрации эффекта Джанибекова.
    \item Демонстрация эффекта Джанибекова.
\end{itemize}

\section{Предварительная теория}
\label{Предварительная теория}
\subsection{Тензор инерции}

Вычислим момент инерции твердого тела относительно произвольной оси ОА. Без ущерба для общности можно принять, что ось проходит через начало координат О (относительно других осей применяем теорему Гюйгенса-Штейнера). 

\begin{wrapfigure}{l}{0.3\linewidth} % l - слева, r - справа, с - центр
\centering
\includegraphics[width=0.5\linewidth]{tenzor.png}
\end{wrapfigure}

Координаты будем обозначать либо $x$, $y$, $z$ либо $x_1$, $x_2$, $x_3$. Разложим радиус-вектор $\vec{r}$ элемента массы тела $dm$ на составляющие вдоль оси ОА и перпендикулярную к ней: $\vec{r} = \vec{r}_\parallel + \vec{r}_\bot$. По определению момента инерции 

\begin{center}
$I = \int\vec{r}_\bot dm = \int (\vec{r} - \vec{r}_\parallel)dm$
\end{center}

Если $\vec{s}$ - единичный вектор вдоль оси ОА, то $\vec{r}_\parallel = (\vec{r}\cdot\vec{s}) = xs_x + ys_y+zs_z$. Кроме того, $r^2 = x^2 + y^2 + z^2, \space \space s^2 = s_x^2 + s_y^2 + s_z^2 = 1$. Получаем
\begin{center}
$I = \int(x^2(1-s_x^2) + y^2(1-s_y^2) + z^2(1-s_z^2) - 2xys_xs_y - 2yzs_ys_z - 2zxs_zs_x)dm = $
\end{center}
\begin{center}
$=\int(s_x^2(y^2 + z^2) + s_y^2(x^2+z^2)+s_z^2(x^2+y^2)-2xys_xs_y - 2yzs_ys_z - 2zxs_zs_x)dm$
\end{center}

Вводя обозначения $I_{xx}, I_{xy}, ...$
\begin{equation} 
\begin{gathered}
\label{(1)}
I = I_{xx}s_{x}^2 + I_{yy}s_{y}^2 + I_{zz}s_{z}^2 + 2I_{xy}s_{x}s_{y} + 2I_{yz}s_{y}s_{z} + 2I_{zx}s_{z}s_{x}
\end{gathered}
\end{equation}

\begin{center}
$I_{xx} = \int_{}^{}(y^2+z^2)dm $ $ I_{xy} \equiv I_{yx} = -\int_{}^{}xydm $ 
\end{center}
\begin{center}
$I_{yy} = \int_{}^{}(z^2+x^2)dm $ $ I_{yz} \equiv I_{zy} = -\int_{}^{}yzdm $ 
\end{center}
\begin{center}
$I_{zz} = \int_{}^{}(x^2+y^2)dm $ $ I_{zx} \equiv I_{xz} = -\int_{}^{}zxdm $ 
\end{center}

Матрица, составленная из этих девяти величин, называется тензором инерции.

\begin{center}
\begin{pmatrix}
I_{xx} & I_{xy} & I_{xz} \\
I_{yx} & I_{yy} & I_{yz} \\
I_{zx} & I_{zy} & I_{zz}
\end{pmatrix}
\end{center}

Так как тензор является симметричным, его можно привести к диагональному виду выбором соответсвующих осей. 

\begin{center}
\begin{pmatrix}
I_{x} & 0 & 0 \\
0 & I_{y} & 0 \\
0 & 0 & I_{z}
\end{pmatrix}
\end{center}

Оси, в которых тензор инерции является диагональным называются главными осями. Если все оси проходят через центр масс тела, то они называются главными центральными осями.

\subsection{Эллипсоид инерции}

Если переобазначить $I_{xx} = I_{11} $, $I_{xy} = I_{12}, ...$, $s_x = s_1, ...$ То формула \ref{(1)} принимает следующий вид

\begin{equation} 
\begin{gathered}
\label{(2)}
I = \sum_{i,j = 1}^{3}I_{ij}s_{i}s_{j}
\end{gathered}
\end{equation}

Формула \ref{(2)} допускает наглядную геометрическую интерпретацию. Через начало координат будем проводить прямые во всевозможных направлениях и на них откладывать отрезки длиной $r = \frac{1}{\sqrt{I}}$. Геометрическим местом таких концов будет некоторая поверхность. Найдём её уравнение. Согласно построению радиус-вектор точки, лежашей на этой поверхности, определяется выражением $\vec{r} = \frac{\vec{s}}{\sqrt{I}}$, а координаты той же точки - $x_i = \frac{s_i}{\sqrt{I}}$ (Здесь использовано переобазначение $x = x_1, y = x_2, z = x_3$). Исключая с помощью этих соотношений величины $s_{i}$ из \ref{(2)}, получим уравнение искомой поверхности 

\begin{equation} 
\begin{gathered}
\label{(3)}
\sum_{i,j=1}^{3}I_{ij}x_ix_j = 1
\end{gathered}
\end{equation}

Эта поверхность второго порядка, очевидно, является эллипсоидом, так как момент инерции $I$, а с ним и длина радиуса-вектора $r$ имеют конечные значения, каково бы ни было направление оси $s$. Она называется эллипсоидом инерции тела относительно точки О, являющейся его центром. При перемещении начала координат О относительно тела будет меняться и эллипсоид инерции тела. Если в качестве О взят центр масс тела, то соответствующий эллипсоид называется центральным. 

В главных осях уравнение эллипсоида инерции принимает упрощённый вид

\begin{equation} 
\begin{gathered}
\label{(4)}
I_xx^2+I_yy^2+I_zz^2 = 1
\end{gathered}
\end{equation}

\subsection{Свободное вращение}

Можно показать, что  во всяком твердом теле существуют три взаимно перпендикулярные оси, совпадающие с главными осями центрального эллипсоида инерции тела, вокруг которых тело может вращаться без воздействия внешних сил. Такие оси называются поэтому свободными или перманентными осями вращения. Последним термином хотят подчеркнуть, что вращение твердого тела по инерции в отсутствие возмущений может продолжаться сколь угодно долго. Иное дело, будет ли это вращение устойчивым по отношению к малым возмущениям, с которыми в реальных условиях всегда надо считаться. Если при наличии таковых характер движения тела меняется мало, т. е. мгновенная ось вращения хотя и непрерывно изменяет свое положение в теле и пространстве, но все время проходит очень близко от соответствующей свободной оси, то вращение вокруг последней будет устойчивым. Если же сколь угодно малое возмущение существенно меняет характер движения тела, т. е. далеко уводит мгновенную ось от исходного направления, то это вращение называется неустойчивым. 

\subsection{Интегралы свободного вращательного движения}
Допустим теперь, что твёрдое тело вращается вокруг какой-то закреплённой или мгновенной оси OA с постоянной или непостоянной угловой скоростью $\omega$. Найдём его момент импульса $\vec{L}$ относительно начала координат O, а также кинетичскую энергию K. По определению 

\begin{center}
    $\vec{L} = \int_{}^{}\vec{r}\times\vec{v} \cdot dm$      
\end{center}

Подставим сюда 
$\vec{v}={\vec{\omega}}\times{\vec{r}}$ и воспользуемся формулой $\vec{r}\times[\;\vec{\omega}\times\vec{r}\;]= r^2\cdot\vec{\omega} - (\vec{\omega}\cdot\vec{r})\cdot\vec{r}$.

\begin{center}
    $\vec{L} = \vec{\omega}\int r^2dm - \int (\;\vec{\omega}\cdot\vec{r}\;)\cdot\vec{r}dm$    
\end{center}

В проекциях на координатные оси это соотношение записывается так:

\begin{center}
$L_x = I_{xx}\omega_x + I_{xy}\omega_y + I_{xz}\omega_z$
\end{center}
\begin{center}
$L_y = I_{yx}\omega_x + I_{yy}\omega_y + I_{yz}\omega_z$
\end{center}
\begin{center}
$L_z = I_{zx}\omega_x + I_{zy}\omega_y + I_{zz}\omega_z$
\end{center}

В системе главных осей формулы упрощаются и принимают вид 

\begin{center}
$L_x = I_{x}\omega_x$
\end{center}
\begin{center}
$L_y = I_{y}\omega_y$
\end{center}
\begin{center}
$L_z = I_{z}\omega_z$
\end{center}

Кинетическая энергия в системе главных осей

\begin{center}
$K = \frac{1}{2}(\vec{L}\cdot\vec{\omega})=\frac{1}{2}(I_x\omega_x^2 + I_y\omega_y^2 + I_z\omega_z^2)$
\end{center}

При свободном вращении момент импульса твердого тела и его кинетическая энергия сохраняются. То есть это и есть интегралы свобдного вращательного движения

\section{Вращение твёрдого тела по инерции вокруг неподвижной точки}

\subsection{Геометрическая интерпретация Луи Пуансо}

Как было сказано раннее выполняются законы схранения энергии и момента импульса. Предварительно возведя уравнение сохранения момента импульса в квадрат получаем 

\begin{center}
$I_x\omega_x^2 + I_y\omega_y^2 + I_z\omega_z^2 = 2K$
\end{center}
\begin{center}
$I_x^2\omega_x^2 + I_y^2\omega_y^2 + I_z^2\omega_z^2 = L^2$
\end{center}

или

\begin{equation} 
\begin{gathered}
\label{(5)}
\frac{L_x^2}{I_x} + \frac{L_y^2}{I_y} + \frac{L_z^2}{I_z} = 2K
\end{gathered}
\end{equation}

\begin{equation} 
\begin{gathered}
\label{(6)}
L_x^2 + L_y^2 + L_z^2 = L^2
\end{gathered}
\end{equation}

Уравнение \ref{(5)} задаёт эллипсоид в осях $L_x, L_y, L_z$ с полуосями $\sqrt{2KI_x}, \sqrt{2KI_y}, \sqrt{2KI_z}$. Уравнение \ref{(6)} задаёт сферу с радиусом L. При перемещении вектора $\vec{L}$ относительно осей инерции волчка его конец движется вдоль линии пересечения указанных поверхностей. На рисунке изображён ряд таких линий пересечения эллипсоида со сферами различных радиусов ($x_1 = x$, $x_2 = y$, $x_3 = z$).

\begin{center}
\includegraphics[width=0.4\linewidth]{1.png}
\end{center}

Если принять момент $I_z$ за максимальный, а $I_x$ за минимальный, то наличие пересечения обеспечивается неравенствами $2KI_x <  L^2 < 2KI_z$, геометрически означающими, что радиус сферы лежит между минимальной и максимальной из полуосей эллипсоида. 

Проследим за изменением характера этих «траекторий» конца вектора $\vec{L}$ (полодии), по мере изменения величины L (при заданной энергии К). Когда $L^2$ лишь немногим превышает $2KI_x$, сфера пересекает эллипсоид по двум замкнутым маленьким кривым, окружающим ось $x$ вблизи соответствующих двух полюсов эллипсоида.

\begin{center}
\includegraphics[width=0.5\linewidth]{min.png}
\end{center}

По мере увеличения $L^2$ кривые расширяются, а при $L^2 = 2KI_y$ превращаются в две кривые, пересекающиеся друг с другом в полюсах эллипсоида на оси $y$.

\begin{center}
\includegraphics[width=0.5\linewidth]{mid.png}
\end{center}

При дальнейшем увеличении $L^2$ до $2KI_z$ вновь возникают две раздельные замкнутые траектории, но окружающие уже полюсы на оси z.

\begin{center}
\includegraphics[width=0.5\linewidth]{max.png}
\end{center}

Замкнутость траектории $\vec{L}$ означает периодичность перемещения вектора $\vec{L}$ по отношению к телу волчка. За время периода вектор $\vec{L}$ описывает некотрую коническую поверхность, возвращаясь в прежнее положение. 

Отметим существенно различный характер траекторий, близких к различным полюсам эллипсоида. Вблизи осей $x$ и $z$ траектории расположены целиком в окрестности полюсов, а траектории, проходящие вблизи полюсов на оси $х_2$, в своем дальнейшем ходе удаляются на большие расстояния от этих точек. Такое различие соответствует разному  характеру устойчивости вращения волчка вокруг ero трех осей инерции. Вращение вокруг осей $x$ и $z$ (отвечающих наибольшему и наименьшему из трех моментов инерции волчка) устойчиво в том смысле, что при малом отклонении от этих состояний волчок будет продолжать совершать движение, близкое к первоначальному. Вращение же вокруг оси $y$ неустойчиво. Достаточно малого отклонения, чтобы возникло движение, уводящее волчок в положения, далекие от первоначального.

В лабораторной системе отсчёта это движение выглядит так:

\begin{center}
\includegraphics[width=0.5\linewidth]{LSO.png}
\end{center}

Это движение и есть геометрическая интерпретация Пуансо. В своих работах он доказал, что вращение тела по инерции около неподвижной точки эквивалентно качению без проскальзывания эллипсоида инерции тела по неизменной плоскости.

\subsection{Исследование уравнениями Эйлера}

Введём вращающуюся систему прямоугольных координат, начало которой совпадает с центром масс тела, а координатные оси направленны вдоль его главных осей. Обозначим через $\vec{i}, \space \vec{j}, \space \vec{k}$ единичные векторы вдоль осей $x, y, z$. Уравнение моментов относительно начала координат запишется в виде 

\begin{equation} 
\begin{gathered}
\label{(7)}
\frac{d}{dt}(I_x\omega_x\cdot\vec{i}+I_y\omega_y\cdot\vec{j}+I_z\omega_z\cdot\vec{k}) = \vec{M}
\end{gathered}
\end{equation}

Здесь подлежат дифференцированию не только компоненты угловой скорости, но и векторы $\vec{i}, \space \vec{j}, \space \vec{k}$. Моменты же инерции $I_x, I_y, I_z$ как величины постоянные дифференцировать не надо. Векторы $\vec{i}, \space \vec{j}, \space \vec{k}$ жёстко связаны с телом, а потому вращаются вместе с ним. Производная $\frac{d\space\vec{i}}{dt}$ означает скорость точки тела, радиус-вектор которой равен \vec{i}. Тогда можно записать

\begin{center}
$\frac{d\space\vec{i}}{dt} = \vec{\omega} \times \vec{i} = \begin{vmatrix}
\vec{i} & \vec{j} & \vec{k} \\
\omega_x & \omega_y & \omega_z \\
1 & 0 & 0
\end{vmatrix} 
= \omega_y\vec{j} - \omega_z\vec{k}$
\end{center}

Аналогично 

\begin{center}
$\frac{d\space\vec{j}}{dt} = \omega_x\vec{k} - \omega_z\vec{i}$, \space
$\frac{d\space\vec{k}}{dt} = \omega_y\vec{i} - \omega_x\vec{j}$
\end{center}

Подставляя в формулу \ref{(7)} получаем

\begin{center}
$
I_x{\displaystyle {\dot {\omega }}_{x}} \vec{i} + I_x\space w_x (\space \omega_z\vec{j} \space -\omega_y \vec{k} \space)+ 
I_y{\displaystyle {\dot {\omega }}_{y}} \vec{j} + I_y\space w_y (\space \omega_x\vec{k} \space -\omega_z \vec{i} \space)+ 
I_z{\displaystyle {\dot {\omega }}_{z}} \vec{k} + I_z\space w_z (\space \omega_y\vec{i} \space -\omega_x \vec{j} \space)
= \vec{M}$
\end{center}

Отсюда 
\begin{center}
$I_x{\displaystyle {\dot {\omega }}_{x}} + \omega_y\omega_z(I_z - I_y) = M_x$
\end{center}
\begin{center}
$I_y{\displaystyle {\dot {\omega }}_{y}} + \omega_z\omega_x(I_x - I_z) = M_y$
\end{center}
\begin{center}
$I_z{\displaystyle {\dot {\omega }}_{z}} + \omega_x\omega_y(I_y - I_x) = M_z$
\end{center}

Это и есть уравнения Эйлера. 

В случае свободного вращения по инерции $\vec{M} = \vec{0}$
Тогда уравнения Эйлера принимают следущий вид:

\begin{equation} 
\begin{gathered}
\label{(8)}
I_x{\displaystyle {\dot {\omega }}_{x}} = \omega_y\omega_z(I_y - I_z)
\end{gathered}
\end{equation}
\begin{equation} 
\begin{gathered}
\label{(9)}
I_y{\displaystyle {\dot {\omega }}_{y}} = \omega_z\omega_x(I_z - I_x) 
\end{gathered}
\end{equation}
\begin{equation} 
\begin{gathered}
\label{(10)}
I_z{\displaystyle {\dot {\omega }}_{z}} = \omega_x\omega_y(I_x - I_y)
\end{gathered}
\end{equation}

Мы предполагаем, что $I_x > I_y > I_z$. 
Рассмотрим ситуацию, когда объект вращается с моментом инерции $I_x$. Для определения характера равновесия предположим, что существуют две малые начальные угловые скорости вдоль других двух осей. В результате, согласно уравнению \ref{(8)}, $\displaystyle {\dot {\omega }}_{x}$ очень мала. Следовательно, зависимостью от времени $\omega_x$ можно пренебречь. Теперь дифференцируем уравнение \ref{(9)} по времени и подставляем $\displaystyle {\dot {\omega }}_{z}$ из уравнения \ref{(10)}

\begin{center}
$I_2I_3\displaystyle {\ddot {\omega }}_{y} = (I_z - I_x)(I_x - I_y)\omega_x^2\omega_y$
\end{center}

Обратим внимание, что знаки у $\omega_y$ и $\displaystyle {\ddot {\omega }}_{y}$ разные, поскольку множитель $(I_z - I_x)$ отрицателен, а множители $(I_x - I_y)$ и $\omega_x^2$ положительны. Следовательно, изначально малая скорость $\omega_y$ будет оставаться малой и в дальнейшем. Аналогично дифференцируя уравнение \ref{(10)} показывается, что знаки у $\omega_z$ и $\displaystyle {\ddot {\omega }}_{z}$ разные, то есть $\omega_z$ также остаётся малой. Тогда и сумма 

\begin{center}
$\vec{\omega}_x + \vec{\omega}_y + \vec{\omega}_z \simeq \vec{\omega}_x$ 
\end{center}

То есть вращение вокруг оси $x$ устойчивое. Аналогично можно показать, что вращение вокруг оси $z$ устойчиво. 
Теперь рассмотрим вращение относительно оси с моментом инерции $I_y$. В этот раз $\displaystyle {\dot {\omega }}_{y}$ очень мала. Следовательно, зависимостью от времени
$\omega_y$ можно пренебречь. Дифференцируем по времени уравнение \ref{(8)} и подставим $\displaystyle {\ddot {\omega }}_{z}$ из уравнения \ref{(10)}

\begin{center}
$I_xI_z{\displaystyle {\ddot {\omega }}_{x}} = (I_y-I_z)(I_x-I_y){\displaystyle {\ddot {\omega }}_{y}}\omega_x$
\end{center}

Знаки у $\omega_x$ и {\displaystyle {\ddot {\omega }}_{x}} одинаковые. Следовательно, изначально малая $\omega_x$ будет возрастать, пока $\displaystyle {\dot {\omega }}_{y}$ не перестанет быть малой и характер вращения вокруг оси $y$ не изменится. То есть даже небольшое возмущение вдоль промежуточной оси сильно уводит ось вращения от первоначального положения. 

\section{Демонстрация эффекта Джанибекова}
\subsection{Эффект Джанибекова}
Проявление теоремы о промежуточной оси в невесомости часто называют эффектом Джанибекова в честь советского космонавта Владимира Джанибекова, который заметил это явление 25 июня 1985 года во время миссии по спасению космической станции «Салют-7». Джанибеков увидел это с гайкой-барашком: скрутив её в невесомости с длинной шпильки, он заметил, что она пролетает немного, разворачивается на 180°, потом, ещё немного пролетев, опять разворачивается. Ссылка на видео \href{https://github.com/sevaphasol/vpv-mechanics/blob/main/%D0%AD%D1%84%D1%84%D0%B5%D0%BA%D1%82%20%D0%94%D0%B6%D0%B0%D0%BD%D0%BD%D0%B8%D0%B1%D0%B5%D0%BA%D0%BE%D0%B2%D0%B0.mp4}{здесь}. На картинке можно видеть траектории главных осей гайки.
\begin{center}
\includegraphics[width=0.4\linewidth]{Jannibekov.jpg}
\end{center}
\par
Хоть эффект и был обнаружен в 1985, его засекретили на 10 лет. Причины этого до сих пор не ясны. Существует конспирологическая теория о том, что наша планета Земля через некоторое время перевернется также как и гайка-барашек и это будет концом света, из-за чего власти боялись публиковать статью об эффекте Джанибекова.  Но эта теория не имеет место быть, так как можно показать, что Земля вращается вокруг оси с максмальным моментом инерции, то есть её вращение устойчиво.
\par
Интересный факт, что в январе 1991 года в журанле <<Journal of Dynamics and Differential Equations>> вышла статья <<The twisting tennis racket>>, которая описывает тот же самый эффект, только на примере подкидывания теннисной ракетки. Ссылка на видео \href{https://github.com/sevaphasol/vpv-mechanics/blob/main/%D0%A2%D0%B5%D0%BD%D0%BD%D0%B8%D1%81%D0%BD%D0%B0%D1%8F%20%D1%80%D0%B0%D0%BA%D0%B5%D1%82%D0%BA%D0%B0.mp4}{здесь}.
\subsection{Устанока для демонстрации}
Установка состоит из мощного фена, трубы и демонстрационного шара.
\begin{center}
\includegraphics[width=0.4\linewidth]{Установка.jpg}
\includegraphics[width=0.4\linewidth]{Шар.jpg}
\end{center}
\par
Из трубы поступает мощный поток воздуха который позволяет находиться шару в почти свободном вращении. Так как шар является не идеально гладким то и момент силы трения воздуха о поверхность шара ненулевой. Однако влияние этого момента очень мало и, что самое главное, нет какого-то выделенного направления. Внутри демонстрационного шара можно видеть куски пластилина с разными массами, красные - около 2 грамм, жёлтые и зелёные - по 1 грамму. Пусть главные центральные оси шара - $X, Y, Z$, причём $I_x < I_y < I_z$. Очевидно, что тогда $X$ проходит через красные куски пластилина, $Y$ через жёлтый и зелёный кусок, а $Z$ проходит через точку пересечения $X$ и $Y$ перпендикулярно плоскости $XY$. То есть вращение около осей $X$ и $Z$ будет устойчивым, а около $Y$ неустойчивым. Как вращается шарик можно посмотреть здесь: \href{https://github.com/sevaphasol/vpv-mechanics/blob/main/%D0%9E%D0%BA%D0%BE%D0%BB%D0%BE%20X.mp4}{около X}, \href{https://github.com/sevaphasol/vpv-mechanics/blob/main/%D0%9E%D0%BA%D0%BE%D0%BB%D0%BE%20Y.mp4}{около Y}, \href{https://github.com/sevaphasol/vpv-mechanics/blob/main/%D0%9E%D0%BA%D0%BE%D0%BB%D0%BE%20Z.mp4}{около Z}. Как видно на втором ролике, зелёный и жёлтый кусок пластилина меняются местами так же, как это делает гайка-барашек в невесомости.

\section{Итоги}
\subsection{Предложения по улучшению}
В первом томе курса теоретической физики Ландау и Лифшица в параграфе §37 <<Ассиметрический волчок>> выведены формулы для вычисления периода обращения промежуточной оси. Используя демонстрационную установку можно провести эксперимент по проверке справедливости этой формулы, оценить влияние силы трения и начального момента (который придается рукой при закручивании шарика). 
\subsection{Вывод}
Эффект Джанибекова - красивое и с первого взгляда парадоксальное явление. Однако помимо красоты у него есть и применение. 
\par
Во-первых, теорема о промежуточной оси занимает самое что ни на есть центральное место в расчетах космических полетов и задействуется в ходе этих расчетов на всех этапах космического полета, начиная от того, как распределить центр масс будущего космического корабля и кончая расчетом расположения антенн и солнечных батарей. Потому что если этого не сделать, то космический корабль начнет вести себя в космосе как гайка Джанибекова. Такое уже случилось с первым американским спутником <<Explorer 1>> еще в начале 60-х, и с тех пор все космические аппараты тщательно рассчитываются на правильный баланс осей.
\par
Во-вторых, эффект Джанибекова нашел применение в квантовой физике. Кванты также обладают угловым моментом, известным как спин. На него можно воздействовать, применяя электромагнитное поле. В работе, опубликованной в журнале <<Scientific Reports>>, ученые обнаружили, что изменения поведения спина можно описать, используя те же математические формулы, которые объясняют теорему вращающейся ракетки. Более подробную информацию можно посмотреть \href{https://www.nature.com/articles/s41598-017-04174-x}{здесь}.

\begin{thebibliography}{9}
\bibitem{landau}
Л.Д. Ландау, Е.М. Лифшиц. Теоретическая физика
МЕХАНИКА
4-е изд., испр. -М.: Наука. Гл. ред. физ.-мат. лит., 1988
\bibitem{sivuhin}
Сивухин Д.В. Общий курс физики. Учеб. пособие: Для вузов. В 5 т. Т.1. Механка. - 4-е изд., стереот. - М.:ФИЗМАТЛИТ; Изд-во МФТИ, 2005 - 560 с.
\end{thebibliography}

\end{document} 
